\documentclass[12pt]{beamer}

\usetheme{Oxygen}
\setbeamertemplate{navigation symbols}{}
\setbeamertemplate{footline}{}

\setbeamercovered{transparent}

\usepackage[utf8]{inputenc}
\usepackage[spanish]{babel}
\usepackage{verbatim}
\usepackage{lgrind}
\usepackage{listings}
\usepackage{beamertexpower}
\usepackage{amsmath}
\usepackage{array}

\pdfinfo
{
  /Title       (QTestLib)
  /Creator     (Kile)
  /Subject     (Testeo de aplicaciones y librerias)
  /Author      (Rafael Fernandez Lopez)
}

\title{QTestLib}
\subtitle{Testeo de aplicaciones y librerías}
\author{Rafael Fernández López}
\institute{}
\date{Marzo 2008}

\lstset{
  basicstyle=\ttfamily,
  showstringspaces=false,
  language=
}

\begin{document}

\frame
{
  \titlepage
  {
    \frametitle{Aditel iParty X}
    \framesubtitle{ereslibre@kde.org}
  }
}

\section*{}
\begin{frame}
  \frametitle{Contenidos}
  \framesubtitle{ereslibre@kde.org}
  \tableofcontents
\end{frame}

\AtBeginSubsection[]
{
  \frame<handout:0>
  {
    \frametitle{Contenidos}
    \tableofcontents[sectionstyle=show/shaded,subsectionstyle=show/shaded]
  }
}

\newcommand<>{\highlighton}[1]{%
  \alt#2{\structure{#1}}{{#1}}
}

\newcommand{\icon}[1]{\pgfimage[height=1em]{#1}}


\section{QTestLib}

\subsection{La importancia de los tests}
\begin{frame}
  \frametitle{QTestLib}
  \framesubtitle{La importancia de los tests}
  \begin{block}{}
    \begin{itemize}
      \item Permiten encontrar errores comunes en la lógica de las librerías y aplicaciones.
      \medskip
      \pause
      \item Aseguran dentro de lo posible que si algo funcionaba, seguirá funcionando, a pesar de haber:
        \begin{itemize}
          \item Introducido nuevas funcionalidades.
          \pause
          \item Modificado funcionalidades existentes.
          \pause
          \item Reestructurado el código de la librería o aplicación internamente.
        \end{itemize}
      \medskip
      \pause
      \item Ayudan a encontrar errores de diseño.
    \end{itemize}
  \end{block}
\end{frame}

\subsection{La infraestructura de Qt}
\begin{frame}
  \frametitle{QTestLib}
  \framesubtitle{La infraestructura de Qt (1)}
  \begin{block}{}
    Según sus diseñadores, QTestLib es/contiene:
    \begin{itemize}
      \item Ligero.
      \pause
      \item Prácticamente independiente.
        \begin{itemize}
          \begin{small}
            \item Únicamente requiere unos pocos símbolos de la librería QtCore para los tests no gráficos.
          \end{small}
        \end{itemize}
      \pause
      \item Rápido y directo.
        \begin{itemize}
          \begin{small}
            \item No requiere aplicaciones externas que ejecuten los tests.
          \end{small}
        \end{itemize}
      \pause
      \item Baterías de datos.
        \begin{itemize}
          \begin{small}
            \item Es posible ejecutar los mismos tests con una grandísima cantidad de datos y casos gracias a la infraestructura.
          \end{small}
        \end{itemize}
      \pause
      \item Tests gráficos.
        \begin{itemize}
          \begin{small}
            \item QTestLib es capaz de simular eventos de ratón y teclado para probar aplicaciones/elementos gráficos.
          \end{small}
        \end{itemize}
    \end{itemize}
  \end{block}
\end{frame}

\begin{frame}
  \frametitle{QTestLib}
  \framesubtitle{La infraestructura de Qt (2)}
  \begin{block}{}
    \begin{itemize}
      \item Integración.
        \begin{itemize}
          \begin{small}
            \item QTestLib emite mensajes que son fácilmente recogidos por entornos de desarrollo como KDevelop o Microsoft Visual Studio.
          \end{small}
        \end{itemize}
      \pause
      \item Thread-safe.
        \begin{itemize}
          \begin{small}
            \item La notificación de errores es atómica y thread-safe.
          \end{small}
        \end{itemize}
      \pause
      \item Seguridad de tipos.
        \begin{itemize}
          \begin{small}
            \item La utilización de templates evita errores introducidos por conversiones de tipo llevadas a cabo implícitamente.
          \end{small}
        \end{itemize}
      \pause
      \item Fácilmente extensible.
        \begin{itemize}
          \begin{small}
            \item Es sencillo añadir nuevos tipos personalizados a los tests y a la salida de los mismos.
          \end{small}
        \end{itemize}
    \end{itemize}
  \end{block}
\end{frame}



\end{document}
