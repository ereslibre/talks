\documentclass[12pt]{beamer}

%\usetheme{Boadilla}
%\usetheme{Madrid}
%\usetheme{Warsaw}
\usetheme{Oxygen}
\setbeamertemplate{navigation symbols}{}
\setbeamertemplate{footline}{}
%\setbeamertemplate{headline}{}

\setbeamercovered{transparent}

\usepackage[utf8]{inputenc}
\usepackage[spanish]{babel}
\usepackage{verbatim}
\usepackage{lgrind}
\usepackage{listings}
\usepackage{beamertexpower}
\usepackage{amsmath}

% COLORES
% \definecolor{minaranja}{HTML}{f7800a}
% \usecolortheme[named=minaranja]{structure}
% FIN COLORES

\pdfinfo
{
  /Title       (Bash)
  /Creator     (TeX)
  /Author      (Rafael Fernández López)
}


\title{Bash}
\subtitle{Bourne Again Shell}
\author{Rafael Fernández López}
\institute{}
\date{}

\lstset{
  basicstyle=\ttfamily,
  showstringspaces=false,
  language=
}

\begin{document}

\frame{\titlepage}

\section*{}
\begin{frame}
  \frametitle{Contenidos}
  \tableofcontents[section=1,hidesubsections]
\end{frame}

\AtBeginSection[]
{
  \frame<handout:0>
  {
    \frametitle{Sección a tratar}
    \tableofcontents[currentsection,hideallsubsections]
  }
}

\AtBeginSubsection[]
{
  \frame<handout:0>
  {
    \frametitle{Subsección a tratar}
    \tableofcontents[sectionstyle=show/hide,subsectionstyle=show/shaded/hide]
  }
}

\newcommand<>{\highlighton}[1]{%
  \alt#2{\structure{#1}}{{#1}}
}

\newcommand{\icon}[1]{\pgfimage[height=1em]{#1}}


\section{Historia e Introducción}

\begin{frame}
  \frametitle{Historia e Introducción}
  \framesubtitle{}

  \begin{itemize}
    \item Iniciado en 1987 por Brian Fox. Mejora cada día gracias a la comunidad del Software Libre
    \medskip
    \pause
    \item Todas las distribuciones de GNU/Linux contienen Bash en su sistema de paquetería
    \medskip
    \pause
    \item Se utiliza en GNU/Linux, FreeBSD, Mac OS X... nativamente. Microsoft Windows cuenta con un entorno de \textit{emulación} llamado Cygwin
    \medskip
    \pause
    \item Algo más que una interfaz de comandos: un lenguaje de programación
  \end{itemize}
  \medskip
  \pause
  \begin{block}{Ejemplo}
    \texttt{cat elquijote.txt | grep `cat filtro.txt` | less}
  \end{block}
\end{frame}


\section{¿Qué es Bash?}

\begin{frame}
  \frametitle{¿Qué es Bash?}
  \framesubtitle{}
  \begin{itemize}
    \item Interfaz de comandos: se encarga de recoger los comandos del usuario para realizar acciones (núcleo y/o aplicaciones)
    \medskip
    \pause
    \item Lenguaje de programación: se pueden escribir guiones que Bash interpretará y ejecutará, con variables, condicionales, bucles...
    \medskip
    \pause
    \item Amigable: Bash admite el coloreado de caracteres. Determinadas aplicaciones o guiones captarán mejor la atención del usuario donde sea necesario
  \end{itemize}
\end{frame}


\section{Mentalidad}

\begin{frame}
  \frametitle{Mentalidad}
  \framesubtitle{}
  \begin{itemize}
    \item GNU/Linux en dos palabras:
    \medskip
    \begin{itemize}
      \item Archivos
      \medskip
      \item Procesos
    \end{itemize}
    \bigskip
    \pause
    \item Programas pequeños, estables y rápidos
    \medskip
    \begin{itemize}
      \item Principio KISS: \textit{Keep It Simple, Stupid}
      \medskip
      \item Cada programa efectúa una tarea muy determinada
      \medskip
      \item Los programas se pueden ``relacionar'' entre sí
      \medskip
      \item Casi cualquier tarea que imaginemos se puede realizar con una correcta combinación de estos pequeños programas
    \end{itemize}
  \end{itemize}

\end{frame}


\section{Ventajas e Inconvenientes}

\begin{frame}
  \frametitle{Ventajas}
  \framesubtitle{}
  \begin{itemize}
    \item Potencia expresiva inmensa
    \medskip
    \pause
    \item Autocompletado
    \medskip
    \pause
    \item Ejecución de tareas secuenciales y/o \textit{paralelas} que pueden ser condicionales, dentro de bucles...
    \medskip
    \pause
    \item Distintos tipos de salida diferenciados (estándar, error...)
    \medskip
    \pause
    \item Consistencia (en GNU/Linux)
    \medskip
    \begin{itemize}
      \item Los dispositivos se tratan como archivos
      \medskip
      \item Sistema de archivos jerarquizado
    \end{itemize}
    \medskip
    \pause
    \item Estabilidad
  \end{itemize}
\end{frame}

\begin{frame}
  \frametitle{Inconvenientes}
  \framesubtitle{}
  \begin{itemize}
    \item Inicialmente menos intuitivo
    \medskip
    \pause
    \item Es necesario \alert{leer}
    \medskip
    \pause
    \item Usuarios con fines de diseño gráfico (retoque de fotografías, diseño en 3D...)
    \medskip
    \begin{itemize}
      \pause
      \item Se pueden realizar algunas operaciones básicas de modificación de imágenes con algunas utilidades \textit{(imagemagick)} destinadas a ello, pero no lo que un diseñador gráfico necesita
      \medskip
      \pause
      \item Se pueden ver películas gracias a librería \textit{libsvga} si se dispone de una tarjeta SVGA, o en su defecto utilizar las librerías que dibujan con caracteres ASCII \textit{(aalib, libcaca...)}
    \end{itemize}
  \end{itemize}
\end{frame}


\section{Uso Básico}

\begin{frame}
  \frametitle{Herramientas de GNU (1)}
  \framesubtitle{}
  \begin{itemize}
    \item \texttt{cd}: cambia al directorio especificado en una ruta absoluta o relativa
    \medskip
    \item \texttt{ls}: lista los archivos y directorios de un determinado directorio
    \medskip
    \item \texttt{cp}: copia uno o varios archivos a un directorio destino
    \medskip
    \item \texttt{mv}: mueve uno o varios archivos a un directorio destino
    \medskip
    \item \texttt{rm}: elimina uno o varios archivos (también puede eliminar directorios enteros con la opción \texttt{-r})
  \end{itemize}
\end{frame}

\begin{frame}
  \frametitle{Herramientas de GNU (1)}
  \framesubtitle{Ejemplos}
  \begin{itemize}
    \item Cambiar al directorio \textit{home}: \texttt{cd $\backsim$}
    \medskip
    \item Listado largo del directorio \textit{home}: \texttt{ls -l $\backsim$}
    \medskip
    \item Listado largo del directorio \textit{home} (mostrando archivos ocultos): \texttt{ls -la $\backsim$}
    \medskip
    \item Copiar archivo.txt en copia.txt: \texttt{cp archivo.txt copia.txt}
    \medskip
    \item Copiar el archivo \texttt{$\backsim$/archivo.txt} en el directorio actual: \texttt{cp $\backsim$/archivo.txt .}
    \medskip
    \item Mover el archivo \texttt{/root/usuarios.txt} al directorio \textit{home}: \texttt{mv /root/usuarios.txt $\backsim$}
    \medskip
    \item Borrar el disco duro: \texttt{rm -rf /}
  \end{itemize}
\end{frame}


\begin{frame}
  \frametitle{Herramientas de GNU (2)}
  \framesubtitle{}
  \begin{itemize}
    \item \texttt{mkdir}: crea un directorio
    \medskip
    \item \texttt{rmdir}: elimina un directorio vacío
    \medskip
    \item \texttt{chown}: modifica el propietario de uno o varios archivos o directorios
    \medskip
    \item \texttt{chgrp}: modifica el grupo de uno o varios archivos o directorios
    \medskip
    \item \texttt{chmod}: modifica los permisos de uno o varios archivos o directorios
  \end{itemize}
\end{frame}

\begin{frame}
  \frametitle{Herramientas de GNU (2)}
  \framesubtitle{Ejemplos}
  \begin{itemize}
    \item Crear el directorio \texttt{proyectos} en el directorio \textit{home}: \texttt{mkdir \$HOME/proyectos}
    \medskip
    \item Elimina el directorio \texttt{proyectos} (\alert{vacío}) del directorio \textit{home}: \texttt{rmdir $\backsim$/proyectos}
    \medskip
    \item Cambia el propietario del archivo \texttt{trabajo.txt} a \textit{juan}: \texttt{chown juan trabajo.txt}
    \medskip
    \item Cambia el grupo del archivo \texttt{fiesta.txt} a \textit{fiesteros}: \texttt{chgrp fiesteros fiesta.txt}
    \medskip
    \item Cambia los permisos a mi directorio \textit{home} (y todos sus contenidos) para que todo el mundo tenga todo tipo de permisos: \texttt{chmod -R 777 $\backsim$}
  \end{itemize}
\end{frame}

\begin{frame}
  \frametitle{Redirecciones (1)}
  \framesubtitle{}
  Existen diferentes descriptores de \textit{ficheros}:
  \medskip
  \begin{itemize}
    \item \texttt{stdin}: entrada estándar, representada por el descriptor número 0
    \medskip
    \item \texttt{stdout}: salida estándar, representada por el descriptor número 1
    \medskip
    \item \texttt{stderr}: salida de error, representada por el descriptor número 2
  \end{itemize}

\end{frame}

\begin{frame}
  \frametitle{Redirecciones (2)}
  \framesubtitle{}
  \begin{itemize}
    \item \texttt{comando > salida.txt}: la salida estándar de \texttt{comando} se escribirá en \texttt{salida.txt} y no por pantalla. Si el archivo contenía algo será sobreescrito
    \medskip
    \item \texttt{comando >> salida.txt}: la salida estándar de \texttt{comando} se añadirá al contenido de \texttt{salida.txt}, y no se mostrará por pantalla
    \medskip
    \item \texttt{comando 2> error.txt}: la salida de error de \texttt{comando} se escribirá en \texttt{error.txt} y no por pantalla
    \medskip
    \item \texttt{comando 2>> error.txt}: la salida de error de \texttt{comando} se añadirá al contenido de \texttt{error.txt}, y no se mostrará por pantalla
  \end{itemize}
\end{frame}

\begin{frame}
  \frametitle{Redirecciones (2)}
  \framesubtitle{Ejemplos}
  \begin{itemize}
    \item \texttt{ls -la > directorioactual.txt}
    \medskip
    \item \texttt{date >> fechasespeciales.txt}
    \medskip
    \item \texttt{ls /root 2> $\backsim$/quefalloocurrio.txt}
    \medskip
    \item \texttt{cp $\backsim$/archivo.txt /root 2>> $\backsim$/logdefallos.txt}
  \end{itemize}
\end{frame}

\begin{frame}
  \frametitle{Redirecciones (3)}
  \framesubtitle{}
  \begin{itemize}
    \item \texttt{comando [redirección] destino 2>\&1}: redirecciona la salida de error de \texttt{comando} a la salida estándar
    \medskip
    \item \texttt{comando [redirección] destino 1>\&2}: redirecciona la salida estándar de \texttt{comando} a la salida de error
    \medskip
    \item \texttt{comando \&> todo.txt}: redirecciona tanto la salida estándar como la de error hacia el fichero \texttt{todo.txt}, sobreescribiendo su contenido, y no se muestra por pantalla
    \medskip
    \item \texttt{comando \&>> todo.txt}: redirecciona tanto la salida estándar como la de error, lo añade al contenido de \texttt{todo.txt} y no se muestra por pantalla
  \end{itemize}
\end{frame}

\begin{frame}
  \frametitle{Redirecciones (3)}
  \framesubtitle{Ejemplos}
  \begin{itemize}
    \item \texttt{ls /root >> $\backsim$/logdefallosyexitos.txt 2>\&1}
    \medskip
    \item \texttt{ls -la > $\backsim$/listado.txt 1>\&2 } \alert{\texttt{listado.txt} vacío}
    \medskip
    \item \texttt{cp -rv $\backsim$ /mnt/usb \&> $\backsim$/backupoperacion.txt}
    \medskip
    \item \texttt{cp -rv $\backsim$ /mnt/usb \&>> $\backsim$/logoperaciones.txt}
  \end{itemize}
\end{frame}

\begin{frame}
  \frametitle{Redirecciones (4)}
  \framesubtitle{}
  Es posible redireccionar la entrada estándar también (\texttt{stdin}). Se puede escribir un comando con la siguiente forma:
  \medskip\linebreak
  \texttt{comando < ficheroConDatos.txt}
  \medskip\linebreak
  De esta forma, \texttt{comando} tomará como datos de entrada el contenido del fichero \texttt{ficheroConDatos.txt}
\end{frame}

\begin{frame}
  \frametitle{Redirecciones (5)}
  \framesubtitle{}
  Los denominados \textit{Here documents} son una manera de pasar datos a un programa de forma que el usuario pueda introducir más de una línea de texto. La sintaxis es la siguiente:
  \medskip\linebreak
  \texttt{comando << SecuenciaDeSalida}\linebreak
  \texttt{> ... datos ...}\linebreak
  \texttt{> ...}\linebreak
  \texttt{> SecuenciaDeSalida}
  \medskip\linebreak
  Características:
  \begin{itemize}
    \item La entrada se va almacenando. Se van creando nuevas líneas pulsando la tecla \textit{Intro}
    \item Se acaban de recibir datos cuando se detecta la cadena de texto que se seleccionó para indicar la salida, en este caso \texttt{SecuenciaDeSalida}
  \end{itemize}
\end{frame}

\begin{frame}
  \frametitle{Redirecciones (6)}
  \framesubtitle{}
  El objetivo de las redirecciones siempre son ficheros. Los \alert{dispositivos son ficheros} en GNU/Linux
  \medskip
  \begin{itemize}
    \item \texttt{comando \&> /dev/null}: tanto la salida estándar como la de error serán redireccionadas al dispositivo \texttt{/dev/null}, que lo único que hace es obviar todo lo que le llegue. De esta forma estamos ``desechando'' la salida de \texttt{comando}, y haciendo que sea una ejecución \textit{silenciosa}
    \medskip
    \item \texttt{comando > /dev/lp}: la salida estándar de \texttt{comando} está siendo redireccionada a la impresora predeterminada
  \end{itemize}
\end{frame}

\begin{frame}
  \frametitle{Tuberías}
  \framesubtitle{}
  \begin{itemize}
    \item \texttt{comando1 | comando2}: la entrada de \texttt{comando2} será tomada de la salida de \texttt{comando1} (que podrá ser salida estándar o de error)
    \medskip
    \item Se pueden encadenar de modo que podemos tener algo así:\medskip\linebreak\texttt{comando1 | comando2 | ... | comando n}
  \end{itemize}
  \begin{block}{Nota}
    Tenemos el mismo resultado si ejecutamos los siguientes dos comandos:
    \begin{itemize}
      \item \texttt{cat archivoConDatos.txt | grep -i ``prueba''} (Tubería)
      \item \texttt{grep -i ``prueba'' < archivoConDatos.txt} (Redirección)
    \end{itemize}
  \end{block}
\end{frame}

\begin{frame}
  \frametitle{Redirecciones y tuberías}
  \framesubtitle{Utilización de GNU \texttt{tee}}
  \texttt{tee} redirecciona la salida estándar tanto a la pantalla como a los ficheros que queramos
  \medskip
  \begin{itemize}
    \item Si queremos redireccionar la salida de error y la estándar es necesario utilizar el tipo de redirección \texttt{``2>\&1''}, ya que \texttt{tee} únicamente trabaja sobre la salida estándar
  \end{itemize}
  \medskip
  El comando \texttt{``ls $\backsim$ | tee mihome.txt''} escribirá la salida estándar del comando \texttt{``ls $\backsim$''} directamente en el archivo \texttt{mihome.txt}, y se mostrará por pantalla.
  \medskip
  \begin{itemize}
    \item Si se desea añadir el resultado del comando al contenido del fichero: \texttt{``ls $\backsim$ | tee -a mihome.txt''}
    \medskip
    \item Para redireccionar \texttt{stdout} y \texttt{stderr} haríamos \texttt{``ls $\backsim$ 2>\&1 | tee mihome.txt''}
  \end{itemize}

\end{frame}

\begin{frame}
  \frametitle{Variables}
  \framesubtitle{}
  \begin{itemize}
    \item Estableciendo una variable:\medskip\linebreak\texttt{export MIS\_DOCUMENTOS=$\backsim$/documentos}
    \medskip
    \item Leyendo el valor de la variable:\medskip\linebreak\texttt{echo \$MIS\_DOCUMENTOS}
    \medskip
    \item Haciendo algo útil con la variable:\medskip\linebreak\texttt{cd \$MIS\_DOCUMENTOS}\medskip\linebreak o más útil todavía:\medskip\linebreak\texttt{cp -r \$PRACTICAS\_COMPI \$MIS\_DOCUMENTOS}
  \end{itemize}
\end{frame}

\begin{frame}
  \frametitle{Flujo de comandos}
  \framesubtitle{}
  Distintos tipos de flujo de comandos:
  \medskip
  \begin{itemize}
    \item Secuencial\medskip\linebreak\texttt{comando1 ; comando2 ; ... ; comando n}
    \medskip
    \item Condicional
    \medskip
    \begin{itemize}
      \item \texttt{Y} lógico:\medskip\linebreak\texttt{comando1 \&\& comando2 \&\& ... \&\& comando n}
      \medskip
      \item \texttt{O} lógico:\medskip\linebreak\texttt{comando1 || comando2 || ... || comando n}
    \end{itemize}
  \end{itemize}
\end{frame}

\begin{frame}
  \frametitle{Comandos anidados}
  \framesubtitle{}
  Con los caracteres especiales \texttt{`} y \texttt{`}, podemos anidar un comando dentro de otro, obteniendo instrucciones y guiones más flexibles.
  \medskip\linebreak
  Con este método podemos tener una instrucción de la forma:
  \medskip\linebreak
  \texttt{comando1 `comando2`}
\end{frame}


\section{Guiones}

\begin{frame}
  \frametitle{¿Qué es un guión?}
  \framesubtitle{}
  Un guión es un conjunto de comandos, sobre los cuales se han podido establecer operaciones de iteración, bifurcación, etc... Permiten:
  \medskip\linebreak
  \begin{itemize}
    \item Realizar acciones repetitivas mucho más rápidamente (múltiples comandos en una sola llamada al guión)
    \medskip
    \item Tomar decisiones en función de los resultados de las operaciones anteriores
    \medskip
    \item Administrar el sistema de una forma más sencilla
  \end{itemize}
\end{frame}

\begin{frame}
  \frametitle{Estructura de un guión}
  \framesubtitle{}
  Para que un guión se pueda ejecutar es necesario:
  \medskip
  \begin{itemize}
    \item Que comience por \texttt{\#!/bin/bash}
    \medskip
    \item Que tenga permisos de ejecución por parte de quien quiera ejecutarlo
  \end{itemize}
  \medskip
  Los símbolos \texttt{\#!} forman un número mágico que indica que es un guión, y que se debe poder ejecutar (si tiene los permisos). En este caso, el \textit{shell} que queremos que lo ejecute es \textit{bash}, cuyo ejecutable se encuentra en \texttt{/bin/bash}.
  \medskip\linebreak
  Se puede seleccionar cualquier otro \textit{shell} para que ejecute el código, pero hay que tener en cuenta que ciertas instrucciones cambian ligeramente de uno a otro. El más usado en general es \textit{bash}.
\end{frame}

\begin{frame}
  \frametitle{Relacionándose con el entorno}
  \framesubtitle{}
  Las formas en las que el guión se relaciona con el \textit{exterior} son de la siguiente manera:
  \medskip
  \begin{itemize}
    \item Variables globales leídas dentro del guión (\texttt{\$PATH}...)
    \medskip
    \item Parámetros pasados al guión
    \begin{itemize}
      \item \texttt{\$1}: parámetro 1; \texttt{\$2}: parámetro 2; ... ; \texttt{\$n}: parámetro n-ésimo
      \item \texttt{\$\#}: obtiene el número de parámetros que han sido pasados el guión
      \item \texttt{\$@}: recoge todos los parámetros que han sido pasados al guión
    \end{itemize}
    \item \texttt{\$?}: pregunta por el código de retorno de la última ejecución de un programa
      \begin{itemize}
        \item 0: éxito
        \item $\neq$0: fallo
      \end{itemize}
    \item Escritura previa en un archivo del que el guión tomará los datos
  \end{itemize}
\end{frame}


\begin{frame}
  \frametitle{Variables en guiones}
  \framesubtitle{}
  Las variables en los guiones se inicializan como cabría esperar. Por ejemplo:
  \medskip\linebreak
  \texttt{i=0} \alert{Sin espacios, o \textit{Bash} no reconocerá la inicialización}
  \medskip\linebreak
  Si en el transcurso del guión queremos cambiar de valor a la variable \texttt{i} existen diferentes sintaxis para hacerlo:
  \medskip
  \begin{itemize}
    \item \texttt{i=\$((i + 1))}
    \item \texttt{i=\$((\$i + 1))}
    \item \texttt{let ``i = i + 1''}
    \item \texttt{let ``i = \$i + 1''}
  \end{itemize}
  \medskip
  \alert{Importante: Bash no puede trabajar con números en coma flotante, únicamente con enteros. Se puede utilizar \texttt{bc}. Algo así: \texttt{echo 3 / 4 | bc -l} devolverá \texttt{0.75}}
\end{frame}

\begin{frame}
  \frametitle{Funciones}
  \framesubtitle{}
  Las funciones nos permiten evitar la repetición de código, y hacen posible la recursividad. Una función en \textit{Bash} se define de la siguiente forma:
  \medskip\linebreak
  \texttt{function mifuncion}\linebreak
  \texttt{\{}\linebreak
  \texttt{...}\linebreak
  \texttt{\}}
  \medskip
  \begin{itemize}
    \item  Dentro de la función se pueden hacer referencia a los parámetros pasados a la función de la forma \texttt{\$1}, para el primer parámetro; \texttt{\$2}, para el segundo parámetro, ... ; \texttt{\$n} para el parámetro n-ésimo
    \medskip
    \item De igual manera, podemos hacer referencia a \texttt{\$\#} para saber el número de parámetros que se le pasó a la función, y también a \texttt{\$@} para obtener la lista de parámetros
  \end{itemize}
\end{frame}

\begin{frame}
  \frametitle{Interacción con el usuario}
  \framesubtitle{}
  Podemos crear menús sencillos gracias a la sentencia \texttt{read}. Por ejemplo:
  \medskip\linebreak
  \texttt{\#!/bin/bash}\linebreak
  \texttt{echo -n ``Escribe tu nombre y apellido: ''}\linebreak
  \texttt{read nombre apellido}\linebreak
  \texttt{echo}\linebreak
  \texttt{echo ``Bienvenid@ \$apellido, \$nombre``}
\end{frame}

\begin{frame}
  \begin{center}
    \begin{Huge}¿Preguntas?\end{Huge}
  \end{center}
\end{frame}

\begin{frame}
  \begin{center}
    \alert{\begin{Huge}¡ Gracias por venir !\end{Huge}}
    \bigskip\linebreak
    http://gul.diskobolo.com
  \end{center}
\end{frame}


\end{document}
