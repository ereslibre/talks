\documentclass[12pt]{beamer}

\usetheme{Oxygen}
\setbeamertemplate{navigation symbols}{}
\setbeamertemplate{footline}{}

\setbeamercovered{transparent}

\usepackage[utf8]{inputenc}
\usepackage[spanish]{babel}
\usepackage{verbatim}
\usepackage{lgrind}
\usepackage{listings}
\usepackage{beamertexpower}
\usepackage{amsmath}
\usepackage{array}

\pdfinfo
{
  /Title       (Goya - Un punto de vista diferente)
  /Creator     (Kile)
  /Subject     (Akademy-es 2007)
  /Author      (Rafael Fernandez Lopez)
}

\title{Goya}
\subtitle{Un punto de vista diferente}
\author{Rafael Fernández López}
\institute{}
\date{Noviembre 2007}

\lstset{
  basicstyle=\ttfamily,
  showstringspaces=false,
  language=
}

\begin{document}

\frame
{
  \titlepage
  {
    \frametitle{Akademy-es 2007}
    \framesubtitle{ereslibre@kde.org}
  }
}

\section*{}
\begin{frame}
  \frametitle{Akademy-es 2007}
  \framesubtitle{ereslibre@kde.org}
  \tableofcontents
\end{frame}

\AtBeginSubsection[]
{
  \frame<handout:0>
  {
    \frametitle{Contenidos}
    \tableofcontents[sectionstyle=show/shaded,subsectionstyle=show/shaded]
  }
}

\newcommand<>{\highlighton}[1]{%
  \alt#2{\structure{#1}}{{#1}}
}

\newcommand{\icon}[1]{\pgfimage[height=1em]{#1}}


\section{Modelo/Vista/Controlador}

\subsection{Conceptos}
\begin{frame}
  \frametitle{Modelo/Vista/Controlador (I)}
  \framesubtitle{Conceptos}

  \begin{block}{}
    \begin{itemize}
      \item El \emph{modelo} ``guarda'' la información en estructuras de datos elegidas para el problema en concreto.
      \medskip
      \pause
      \item La \emph{vista} decide dónde se dibujará cada elemento y con qué opciones, a veces dependiendo de ciertas propiedades que el modelo conoce. Recorrerá al menos todos los elementos que necesitan ser dibujados estableciendo sus propiedades (posición, estado, ...).
      \medskip
      \pause
      \item El \emph{controlador} se encarga de dibujar un elemento, ignorando todo lo que le rodea.
    \end{itemize}
  \end{block}

\end{frame}

\subsection{Ventajas}
\begin{frame}
  \frametitle{Modelo/Vista/Controlador (II)}
  \framesubtitle{Ventajas}

  \begin{block}{}
    \begin{itemize}
      \item Favorece a la óptima utilización de memoria. No se desperdician recursos.
        \begin{itemize}
          \item \emph{Modelo} compartido, \emph{controlador} específico.
          \medskip
          \pause
          \item \emph{Controlador} compartido, \emph{modelo} específico.
          \medskip
          \pause
          \item Otras posibilidades.
        \end{itemize}
      \medskip
      \pause
      \item Fácilmente extendible.
      \medskip
      \pause
      \item Favorece la abstracción \textcolor{oxygenorange}{$\rightarrow$} Simplifica la reutilización de código.
    \end{itemize}
  \end{block}

\end{frame}

\subsection{Desventajas}
\begin{frame}
  \frametitle{Modelo/Vista/Controlador (III)}
  \framesubtitle{Desventajas}

  \begin{block}{}
    \begin{itemize}
      \item Interacción con el usuario.
        \begin{itemize}
            \item Editores.
            \begin{itemize}
              \item \alert{No} es su finalidad.
              \medskip
              \pause
              \item Gasto de memoria masivo e innecesario, contrario a la ideología \emph{MVC}.
              \medskip
              \pause
              \item Anomalías fruto de un uso incorrecto.
              \medskip
              \pause
              \item Potencia muy reducida, al menos para versiones de Qt anteriores a la 4.4.
            \end{itemize}
        \end{itemize}
      \medskip
      \pause
      \item Accesibilidad.
    \end{itemize}
  \end{block}

\end{frame}

\section{Goya}

\subsection{Misión}
\begin{frame}
  \frametitle{Goya (I)}
  \framesubtitle{Misión}

  \begin{block}{}
    \begin{itemize}
      \item Permitir la interacción con el usuario.
      \begin{itemize}
        \item No complicar la interfaz \emph{MVC}.
        \medskip
        \pause
        \item Seguir las directrices \emph{MVC} salvando la mayor cantidad de memoria posible.
        \medskip
        \pause
        \item Mayor versatilidad posible.
        \medskip
        \pause
        \item Uso de \emph{signals} y \emph{slots}.
      \end{itemize}
      \medskip
      \pause
      \item Mejorar la accesibilidad.
    \end{itemize}
  \end{block}

\end{frame}

\subsection{Estructura}
\begin{frame}
  \frametitle{Goya (II)}
  \framesubtitle{Estructura}

  \begin{block}{}
    \begin{itemize}
      \item Capa adicional que se sitúa entre la \emph{vista} y el \emph{controlador}, capturando eventos y reaccionando ante ellos.
    \end{itemize}
    \medskip
    \pause
    \begin{center}
      \begin{tabular}{c c c c c}
        \emph{Controlador} & \textcolor{oxygenorange}{$\rceil$} \\
        \textcolor{oxygenorange}{$\Updownarrow$} \\
        Goya & \textcolor{oxygenorange}{$\Leftrightarrow$} & \emph{Modelo} & \textcolor{oxygenorange}{$\Leftrightarrow$} & Datos\\
        \textcolor{oxygenorange}{$\Updownarrow$} \\
        \emph{Vista} & \textcolor{oxygenorange}{$\rfloor$}
      \end{tabular}
    \end{center}
    \begin{tiny}\begin{center}Donde ``\textcolor{oxygenorange}{$\Leftrightarrow$}'' indica flujo de información.\end{center}\end{tiny}
  \end{block}
\end{frame}

\end{document}
