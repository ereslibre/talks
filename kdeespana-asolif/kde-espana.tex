\documentclass[12pt]{beamer}

\usetheme{Air}
\usepackage[spanish]{babel}
\usepackage{thumbpdf}
\usepackage{wasysym}
\usepackage{ucs}
\usepackage[utf8]{inputenc}
\usepackage{pgf,pgfarrows,pgfnodes,pgfautomata,pgfheaps,pgfshade}
\usepackage{verbatim}
\usepackage{ulem}
\usenavigationsymbolstemplate{}

\pdfinfo
{
  /Title       (KDE-ESPANA)
  /Author      (RAFAEL FERNANDEZ LOPEZ)
}

\AtBeginSection[]
{
  \frame<handout:0>
  {
    \frametitle{Encuentro ASOLIF 2010}
    \tableofcontents[currentsection]
  }
}

\title{KDE España}
\subtitle{Una asociación para \textbf{TÍ}}
\author{Rafael Fernández López}
\date{Abril, 2010}

\begin{document}

\begin{frame}{Encuentro ASOLIF 2010}
  \framesubtitle{ereslibre@kde.org}
  \titlepage
\end{frame}


\section*{}
\begin{frame}{Encuentro ASOLIF 2010}
  \framesubtitle{}
  \tableofcontents[section=1]
\end{frame}

\newcommand<>{\highlighton}[1]{%
  \alt#2{\structure{#1}}{{#1}}
}

\newcommand{\icon}[1]{\pgfimage[height=1em]{#1}}


\section{Un poco de historia}

\begin{frame}{KDE en España}
	\framesubtitle{Kaixo! Eu parlo castellano}
	\begin{itemize}
		\item 25 de Agosto de 1997
		\begin{itemize}
			\item Revisión 849 - [Roberto] Spanish translation for KDE. I am not really sure if the code
should be es or sp. Anybody knows?
		\end{itemize}
 		\item 16 de Marzo de 1998
		\begin{itemize}
 			\item Revisión 6144 - added catalan
		\end{itemize}
		\item 31 de Agosto de 2000
		\begin{itemize}
			\item Revisión 62283 - Added again the updated desktop and kdelibs .po files (Gallego)
		\end{itemize}
		\item 21 de Marzo de 2001
		\begin{itemize}
 			\item Revisión 88089 - *** empty log message *** (Euskera)
		\end{itemize}
	\end{itemize}
\end{frame}

\begin{frame}{KDE en España}
	\framesubtitle{Los programadores}
	\begin{itemize}
		\item XXX de 1997 (beta de KDE 0.12) - Antonio Larrosa
 		\item Julio de 2003 - Albert Astals Cid
		\item Marzo de 2006 - Alfredo Beaumont Sainz
		\item Octubre de 2006 - Rafael Fernández López
		\item Febrero de 2007 - Aleix Pol i Gonzàlez
		\item Noviembre de 2007 - Eduardo Robles Elvira
		\item Diciembre de 2008 - Alex Fiestas
	\end{itemize}
\end{frame}

\section{¿Qué es KDE España?}

\begin{frame}{¿Qué es KDE España?}
  \framesubtitle{}
  \begin{itemize}
    \item KDE España es una \textbf{asociación cultural} sin ánimo de lucro
    \item Cuenta con una junta directiva de \textbf{cuatro} personas:
    \begin{itemize}
      \item Presidente - Albert Astals Cid
      \item Vicepresidente - Rafael Fernández López
      \item Tesorero - Ana Beatriz Guerrero López
      \item Secretario - Aleix Pol i Gonzàlez
    \end{itemize}
    \item De momento no contamos con vocales
    \item Cuenta con \textbf{23} asociados a día de hoy
    \item Formada el 9 de julio de \textbf{2009} en \textbf{Las Palmas} de Gran Canaria
  \end{itemize}
\end{frame}

\section{¿Qué NO es KDE España?}

\begin{frame}{¿Qué NO es KDE España?}
  \framesubtitle{}
  \begin{itemize}
    \item Un \textbf{competidor} del \textbf{KDE e.V.}
    \begin{itemize}
      \item Persigue exactamente los \textbf{mismos objetivos} que el KDE e.V.
      \item Asociación ``\textbf{hermana}''
      \item \textbf{Independiente} en cuanto a recursos
      \begin{itemize}
        \item No depende del KDE e.V., cuenta con sus propios recursos
      \end{itemize}
      \item Un \textbf{conjunto} de personas \textbf{determinado}: usuarios y
            contribuidores de nacionalidad española o residentes en el territorio estatal.
    \end{itemize}
    \item Un \textbf{representante legal} del proyecto \textbf{KDE}
  \end{itemize}
\end{frame}


\section{Objetivos de KDE España}

\begin{frame}{Objetivos de KDE España (1)}
  \framesubtitle{}
  \begin{itemize}
    \item \textbf{Promover} la utilización de \textbf{software libre}, y en concreto del entorno de escritorio \textbf{KDE}
    \item \textbf{Desarrollar} y \textbf{mejorar} KDE como entorno de escritorio y entorno de desarrollo basado en software libre.
    \begin{itemize}
      \item Aportaciones de \textbf{código}
      \item Aportaciones \textbf{artísticas}
      \item \textbf{Traducciones}
      \item \textbf{Documentación}
      \item ...
    \end{itemize}
    \item \textbf{Impulsar} el \textbf{crecimiento} de la comunidad de \textbf{desarrolladores} y \textbf{colaboradores} de KDE en España
    \begin{itemize}
      \item Actividades de \textbf{generación de conocimiento}
    \end{itemize}
  \end{itemize}
\end{frame}

\begin{frame}{Objetivos de KDE España (2)}
  \framesubtitle{}
  \begin{itemize}
    \item \textbf{Colaborar} y \textbf{animar a otros} a colaborar en KDE
    \item \textbf{Facilitar} la \textbf{comunicación} entre usuarios y desarrolladores de KDE
    \item \textbf{Promover} la \textbf{traducción} y \textbf{adaptación} de KDE a las distintas lenguas del estado español
    \item \textbf{Promover} y \textbf{apoyar} la realización de \textbf{actividades o encuentros} de colaboradores y usuarios de KDE
    \item \textbf{Colaborar} e \textbf{intercambiar conocimientos} con proyectos similares a nivel nacional e internacional
    \item \textbf{Difundir} el conocimiento y \textbf{facilitar} el aprendizaje de las tecnologías del entorno de escritorio KDE a los interesados en participar en la comunidad
  \end{itemize}
\end{frame}

\begin{frame}{Objetivos de KDE España (3)}
  \framesubtitle{}
  \begin{itemize}
    \item \textbf{Dinamizar} el \textbf{contacto} entre las empresas, instituciones oficiales y la comunidad de usuarios y colaboradores en cuestiones relacionadas con KDE
  \end{itemize}
\end{frame}

\begin{frame}{¿Cómo me apunto?}
    \begin{block}{Artículo 27}
        Para poder ser admitido como socio de número se ha de obtener la nominación de un socio fundador o de número así como el apoyo de otros dos socio fundadores o de número. Una vez obtenidos los apoyos necesarios se abrirá un período de discusión de una semana seguida por una votación de dos semanas. El aspirante a socio de número deberá obtener la mayoría simple en dicha votación. Esta votación será realizada por los miembros fundadores y de número.
    \end{block}
\end{frame}

\frame{
  \frametitle{Encuentro ASOLIF 2010}
  \vspace{1.5cm}
  {\huge \alert{\textbf{Gracias.}} ¿Preguntas?}

  \vspace{1cm}
  \begin{center}
    \large \textbf{http://www.kde-espana.es}
  \end{center}


  \vspace{1cm}
  \begin{flushright}
    Rafael Fernández López

    \structure{\footnotesize{ereslibre@kde.org}}
  \end{flushright}
}


\end{document}
